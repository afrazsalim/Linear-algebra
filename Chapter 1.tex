\documentclass{article}
\usepackage{amsmath}
\usepackage{bbold}
\begin{document}
\title{Chapter 1 : Eerstegraadsvergelikingen}
\author{Afraz Salim}
\maketitle
\section{Bepaal of de uitspraken waar of vals zijn en verklaar je antwoorden}
\begin{itemize}
\item Als A en B matrixs zijn zodat $AB = O$ dan is $A = 0$ of $B = 0$.
\newline
Antwoord: vals.
\newline
Tegenvoorbeeld: $A = 
\begin{bmatrix}
1 &-1 \\
-1&  1\\
\end{bmatrix}
$
$ B = 
\begin{bmatrix}
1 & 1\\
1 & 1\\
\end{bmatrix}
$
Nu $A*B = 0$ maar $A  \neq 0$ nog $B = 0$.

\item Als $A ^2 = \mathbb{1}$ en $B^2 = \mathbb{1}$ dan $(AB)^-1 = BA$.
\newline
Antwoord: waar.
\newline
Wij vermenigvuldigen beide zijden met $AB$ zodat $AB(AB)^-1 = AB^2A $\newline
$AB(AB)^-1 = \mathbb{1}$ en $A(B^2 = \mathbb{1}$ Dit is al gegeven $) A$ \newline
$\mathbb{1} = A*\mathbb{1}*A$\newline
$\mathbb{1} = A^2$\newline
$\mathbb{1} = \mathbb{1}$\newline

\item  Als $A$ en $B$ inverteerbaar matrices zijn dan is $A + B$ ook inverteerbaar.\newline
Antwoord: vals\newline
Tegenvoorbeeld:
   $A = \begin{bmatrix}
   1 & 0\\
   0 & 1\\
   \end{bmatrix}$
   
   $B = \begin{bmatrix}
   -1 & 0\\
    1 & 1\\
   \end{bmatrix}
   $
A en B zijn inverteerbaar maar niet $A+B = 0$.
\item If $A$ en $B$ en $AB$ symmetrisch zijn dan is $AB = BA$.\newline
 Antwoord waar.\newline
 Proof: $AB = (AB)^T$ omdat $AB$ Symmetrisch is dan $AB = B^T*A^T = BA$.

\item Als $A$ en $B$ zijn symmetrisch dan is $AB$ ook symmetrisch.\newline
Antwoord: waar
\newline
Proof:
Omdat $A$ en $B$ symmetrisch zijn ,$a_{ij} = a_{ji}$ en $b_{ij} = b_{ji}$.\newline
Let $c_{ij}  = a_{ij}*b_{ij}$ en $c_{ji} = a_{ji}*b_{ji}$ dan $c_{ij} = c_{ji}$ Symmetrisch.

\item Als $A$ is niet inverteerbaar dan is ook $AB$ niet inverteerbaar.
\newline
Antwoord waar:\newline
$f(A.B) = f(A).f(B)$ en indien $f(A) = 0$ dan voor elke $B$ is $AB = 0$.\newline
\item Als $E_1$ en $E_2$ twee elementaire matrices zijn dan is $E_1*E_2 = E_2*E_1$.
\newline
Antwoord: Niet waar.
Tegenvoorbeeld:
  $A = \begin{bmatrix}
  1 & 2\\
  0 & 1\\
  \end{bmatrix}
 $
$B = \begin{bmatrix}
1 & 0\\
3 & 1\\
\end{bmatrix}$
\newline
$A*B \neq B*A$.
\newline

\item Exercise $21$:\newline
$A*A^T$ en $A+ A^T$ zijn symmetrisch.
\newline
Antwoord:waar
\newline
$C = A*A^T$ dan $c_{ij} = a_{ij}*(a_{ji})^T  ==> c_{ij} = a_{ij}*a_{ij}$  .\newline
Dus $c_{ij}$ is symmetrisch want elk element wordt met zichzelf vermigvuldigd.
\newline
\item $A-A^T$ is scheefsymmetrisch.
Antwoord:Waar \newline
Een matric is scheefsymmetrisch indien $A^T = -A$.\newline
$(A-A^T)^T = -(A-A^T)$ \newline
$A^T - A = -A + A^T$\newline
$A^T - A =  -A + A^T$\newline
Proved.
\newline
Nilpoten: Een matrix is niet-nilpoten indien voor elke $k$ $A^k \neq 0$.\newline
\item Toon aan dat een inverteerbare matrix nipotent is.
\newline
Stel dat een matrix niet-nilpoten is dan is die matrix inverteerbaar en voor elke $k$ $A^k $ is hetzelfda als een A matrix $k$ keer met zichzelf vermenigvuldigen.\newline
TODO :(


\item Bepaal Lu-decompositie van een matrix:\newline
$A = \begin{bmatrix}
1 & 0 & 0\\
0 & 1 & 0\\
0 & 0 & 1\\
\end{bmatrix}
|
\begin{bmatrix}
1 & -1 & 0\\
-1 & 2 & -1\\
0 & -1 & 2\\
\end{bmatrix}\overset{r_1+r_2}{\longrightarrow} 
$

$\overset{R2 -> R2+R1}{\longrightarrow}\begin{bmatrix}
1 & 0 & 0\\
1 & 1 & 0\\
0 & 0 & 1\\
\end{bmatrix}
|
\begin{bmatrix}
1 & -1 & 0\\
0 & 1 & -1\\
0 & -1 & 2\\
\end{bmatrix} $


$\overset{R3 -> R3+R2}{\longrightarrow}\begin{bmatrix}
1 & 0 & 0\\
0 & 1 & 0\\
0 & 1 & 1\\
\end{bmatrix}
|
\begin{bmatrix}
E_1
\end{bmatrix}
\begin{bmatrix}
1 & -1 & 0\\
0 & 1 & -1\\
0 & 0 & 1\\
\end{bmatrix} $
\newline
$A =LU$ dus $A = {E_1}^{-1}*{E_2}^{-1}*U$
\newline
$E_1^{-1} = \begin{bmatrix}
1 & 0 & 0\\
1 & 1 & 0\\
0 & -1 & 1\\
\end{bmatrix}$
\newline
$E_2^{-1} = \begin{bmatrix}
1 & 0 & 0\\
-1 & 1 & 0\\
0 & 0 & 1\\
\end{bmatrix}$
\newline

$E_2^{-1}*E_1{-1} = L$
\end{itemize}
\end{document}